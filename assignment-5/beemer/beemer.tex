%%%%%%%%%%%%%%%%%%%%%%%%%%%%%%%%%%%%%%%%%%%%%%%%%%%%%%%%%%%%%%%
%
% Welcome to Overleaf --- just edit your LaTeX on the left,
% and we'll compile it for you on the right. If you open the
% 'Share' menu, you can invite other users to edit at the same
% time. See www.overleaf.com/learn for more info. Enjoy!
%
%%%%%%%%%%%%%%%%%%%%%%%%%%%%%%%%%%%%%%%%%%%%%%%%%%%%%%%%%%%%%%%


% Inbuilt themes in beamer
\documentclass{beamer}

% Theme choice:
\usetheme{CambridgeUS}

% Packages
\usepackage{graphicx}
\graphicspath{{images/}}
\usepackage{gensymb}
\usepackage{amssymb}
\usepackage[cmex10]{amsmath}
\usepackage{amsthm}
\usepackage[export]{adjustbox}
\usepackage{bm}
\usepackage{longtable}
\usepackage{enumitem}
\usepackage{mathtools}
 \usepackage{tikz}
\usepackage[breaklinks=true]{hyperref}
\usepackage{listings}
\usepackage{color}                                            %%
\usepackage{array}                                            %%
\usepackage{longtable}                                        %%
\usepackage{calc}                                             %%
\usepackage{multirow}                                         %%
\usepackage{hhline}                                           %%
\usepackage{ifthen}                                           %%
\usepackage{lscape}     
\usepackage{multicol}
% \usepackage{enumerate}
\DeclareMathOperator*{\Res}{Res}
\renewcommand\thesection{\arabic{section}}
\renewcommand\thesubsection{\thesection.\arabic{subsection}}
\renewcommand\thesubsubsection{\thesubsection.\arabic{subsubsection}}
\renewcommand\thesectiondis{\arabic{section}}
\renewcommand\thesubsectiondis{\thesectiondis.\arabic{subsection}}
\renewcommand\thesubsubsectiondis{\thesubsectiondis.\arabic{subsubsection}}
\hyphenation{op-tical net-works semi-conduc-tor}
\def\inputGnumericTable{}                                 %%
\lstset{
frame=single, 
breaklines=true,
columns=fullflexible
}

\newcommand{\BEQA}{\begin{eqnarray}}
\newcommand{\EEQA}{\end{eqnarray}}
\newcommand{\define}{\stackrel{\triangle}{=}}
\newcommand*\circled[1]{\tikz[baseline=(char.base)]{
    \node[shape=circle,draw,inner sep=2pt] (char) {#1};}}
%\bibliographystyle{IEEEtran}
\providecommand{\mbf}{\mathbf}
\providecommand{\pr}[1]{\ensuremath{\Pr\left(#1\right)}}
\providecommand{\qfunc}[1]{\ensuremath{Q\left(#1\right)}}
\providecommand{\sbrak}[1]{\ensuremath{{}\left[#1\right]}}
\providecommand{\lsbrak}[1]{\ensuremath{{}\left[#1\right.}}
\providecommand{\rsbrak}[1]{\ensuremath{{}\left.#1\right]}}
\providecommand{\brak}[1]{\ensuremath{\left(#1\right)}}
\providecommand{\lbrak}[1]{\ensuremath{\left(#1\right.}}
\providecommand{\rbrak}[1]{\ensuremath{\left.#1\right)}}
\providecommand{\cbrak}[1]{\ensuremath{\left\{#1\right\}}}
\providecommand{\lcbrak}[1]{\ensuremath{\left\{#1\right.}}
\providecommand{\rcbrak}[1]{\ensuremath{\left.#1\right\}}}
\theoremstyle{remark}
\newtheorem{rem}{Remark}
\newcommand{\sgn}{\mathop{\mathrm{sgn}}}
%\providecommand{\abs}[1]{\left\vert#1\right\vert}
%\providecommand{\res}[1]{\Res\displaylimits_{#1}} 
%\providecommand{\norm}[1]{\left\lVert#1\right\rVert}
%\providecommand{\norm}[1]{\lVert#1\rVert}
%\providecommand{\mtx}[1]{\mathbf{#1}}
%\providecommand{\mean}[1]{E\left[ #1 \right]}
\providecommand{\fourier}{\overset{\mathcal{F}}{ \rightleftharpoons}}
%\providecommand{\hilbert}{\overset{\mathcal{H}}{ \rightleftharpoons}}
\providecommand{\system}{\overset{\mathcal{H}}{ \longleftrightarrow}}
	%\newcommand{\solution}[2]{\textbf{Solution:}{#1}}
\newcommand{\solution}{\noindent \textbf{Solution: }}
\newcommand{\cosec}{\,\text{cosec}\,}
\providecommand{\dec}[2]{\ensuremath{\overset{#1}{\underset{#2}{\gtrless}}}}
\newcommand{\myvec}[1]{\ensuremath{\begin{pmatrix}#1\end{pmatrix}}}
\newcommand{\mydet}[1]{\ensuremath{\begin{vmatrix}#1\end{vmatrix}}}
\newcommand*{\permcomb}[4][0mu]{{{}^{#3}\mkern#1#2_{#4}}}
\newcommand*{\perm}[1][-3mu]{\permcomb[#1]{P}}
\newcommand*{\comb}[1][-1mu]{\permcomb[#1]{C}}
\numberwithin{equation}{subsection}
\makeatletter
\@addtoreset{figure}{problem}
\makeatother
\let\StandardTheFigure\thefigure
\let\vec\mathbf
\renewcommand{\thefigure}{\theproblem}
\def\putbox#1#2#3{\makebox[0in][l]{\makebox[#1][l]{}\raisebox{\baselineskip}[0in][0in]{\raisebox{#2}[0in][0in]{#3}}}}
     \def\rightbox#1{\makebox[0in][r]{#1}}
     \def\centbox#1{\makebox[0in]{#1}}
     \def\topbox#1{\raisebox{-\baselineskip}[0in][0in]{#1}}
     \def\midbox#1{\raisebox{-0.5\baselineskip}[0in][0in]{#1}}
\vspace{3cm}


% Title page details: 
\title{Probability Exercise 13.5, Q2} 
\author{Hema Sri Cheekatla, CS21BTECH11013}
\date{\today}
\logo{\large \LaTeX{}}


\begin{document}
% Title page frame
\begin{frame}
    \titlepage 
\end{frame}

% Remove logo from the next slides
\logo{}


% Outline frame
\begin{frame}{Outline}
    \tableofcontents
\end{frame}

% Question frame
\begin{frame}{Question}
\section{Question}
	A pair of dice is thrown 4 times. If getting a doublet is considered a success, find
the probability of two successes.
\end{frame}

% Solution frames
\begin{frame}{Solution}
\section{Solution}
On throwing a pair of dice, getting doublet is success and we are throwing the pair of dice 4 times. \newline
\textbf{Doublets: (1,1), (2,2), (3,3), (4,4), (5,5), (6,6)} \newline
Consider one event of getting success represented by a Bernoulli random variable X as shown 
\begin{table}[h]
	\caption{Events for X}
	\label{tab:mesh}
	\input{tables/table1.tex}
\end{table}
\end{frame}

\begin{frame}
\begin{align}
	P_X(1) &= \frac{1}{6} \\
	P_X(0) &= \frac{5}{6} 
\end{align}
\end{frame}

\begin{frame}{}
Now consider another event of throwing this pair of dice 4 times and represent the corresponding outcomes by the Binomial Random Variable Y as shown
\begin{table}[h]
	\caption{Events for Y}
	\label{tab:mesh}
	\input{tables/table2.tex}
\end{table}
\end{frame}

\begin{frame}{}
The corresponding Probabilities is given by,
\begin{align}
	P_Y(k) &= \begin{cases} \comb{n}{k} p^k q^{n-k} , & k = 1,2,\dots ,n \\
	0 , & otherwise \end{cases}
\end{align}
Where n is number of trails = 4\newline
p = $P_X(1) = \frac{1}{6}$ \newline
q = $P_X(0) = \frac{5}{6}$ \newline
\end{frame}

\begin{frame}
Hence the probability of getting success twice is given by,
\begin{align}
	P_Y(2) &= \comb{4}{2} \brak{\frac{1}{6}}^2 \brak{\frac{5}{6}}^{4-2} \\
	P_Y(2) &= \frac{25}{216}
\end{align}
\begin{block}{}
Hence the required probability is $\frac{25}{216}$
\end{block}
\end{frame}
\begin{frame}
	Now the Probability Mass Function of Y is as follows
\begin{figure}[h]
	\includegraphics[width = 0.5\textwidth]{plot}
	\label{fig:mesh}
	\caption{Probability Mass Function of Y}
\end{figure}
\end{frame}

\end{document}
