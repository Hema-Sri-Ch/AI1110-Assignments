%%%%%%%%%%%%%%%%%%%%%%%%%%%%%%%%%%%%%%%%%%%%%%%%%%%%%%%%%%%%%%%
%
% Welcome to Overleaf --- just edit your LaTeX on the left,
% and we'll compile it for you on the right. If you open the
% 'Share' menu, you can invite other users to edit at the same
% time. See www.overleaf.com/learn for more info. Enjoy!
%
%%%%%%%%%%%%%%%%%%%%%%%%%%%%%%%%%%%%%%%%%%%%%%%%%%%%%%%%%%%%%%%


% Inbuilt themes in beamer
\documentclass{beamer}

% Theme choice:
\usetheme{CambridgeUS}

 \usepackage{gensymb}
 \usepackage{polynom}
\usepackage{amssymb}
\usepackage[cmex10]{amsmath}
\usepackage{amsthm}
 \usepackage{stfloats}
 \usepackage{bm}
 \usepackage{longtable}
\usepackage{enumitem}
 \usepackage{mathtools}
 \usepackage{tikz}
\usepackage[breaklinks=true]{hyperref}
\usepackage[latin1]{inputenc}  
    \usepackage{color}                                            %%
    \usepackage{array}                                            %%
    \usepackage{longtable}                                        %%
    \usepackage{calc}                                             %%
    \usepackage{multirow}                                         %%
    \usepackage{hhline}                                           %%
    \usepackage{ifthen}                                           %%

\def\inputGnumericTable{}



\newcommand{\BEQA}{\begin{eqnarray}}
\newcommand{\EEQA}{\end{eqnarray}}
\newcommand{\define}{\stackrel{\triangle}{=}}
\newcommand*\circled[1]{\tikz[baseline=(char.base)]{
    \node[shape=circle,draw,inner sep=2pt] (char) {#1};}}

\providecommand{\mbf}{\mathbf}
\providecommand{\pr}[1]{\ensuremath{\Pr\left(#1\right)}}
\providecommand{\qfunc}[1]{\ensuremath{Q\left(#1\right)}}
\providecommand{\sbrak}[1]{\ensuremath{{}\left[#1\right]}}
\providecommand{\lsbrak}[1]{\ensuremath{{}\left[#1\right.}}
\providecommand{\rsbrak}[1]{\ensuremath{{}\left.#1\right]}}
\providecommand{\brak}[1]{\ensuremath{\left(#1\right)}}
\providecommand{\lbrak}[1]{\ensuremath{\left(#1\right.}}
\providecommand{\rbrak}[1]{\ensuremath{\left.#1\right)}}
\providecommand{\cbrak}[1]{\ensuremath{\left\{#1\right\}}}
\providecommand{\lcbrak}[1]{\ensuremath{\left\{#1\right.}}
\providecommand{\rcbrak}[1]{\ensuremath{\left.#1\right\}}}

\makeatletter
\@addtoreset{figure}{problem}
\makeatother
\let\StandardTheFigure\thefigure
\let\vec\mathbf
% Title page details: 
\title{Probability, Exercise 16.3, Q5} 
\author{Hema Sri Cheekatla, CS21BTECH11013}
\date{\today}
\logo{\large \LaTeX{}}


\begin{document}

% Title page frame
\begin{frame}
    \titlepage 
\end{frame}

% Remove logo from the next slides
\logo{}


% Outline frame
\begin{frame}{Outline}
    \tableofcontents
\end{frame}


% Lists frame
\section{Question}
\begin{frame}{Question}
A fair coin with 1 marked on one face and 6 on the other and a fair die are both tossed. Find the probability that the sum of the numbers that turn up is 
\begin{itemize}
    \item[o] 3
    \item[o] 12
\end{itemize}

\end{frame}

\begin{frame}{Solution:}
The coin can turn up either 1 or 6 only, where as die can turn up 1, 2, 3, 4, 5, 6\newline
As well as the coin and the die both are fair, hence the chance of getting any number is equally probable for both coin and die.\newline
i.e., \newline
$ \pr{A = i} = \frac{1}{2} $ , where i = 1, 6; \newline
$ \pr{A = i} = \frac{1}{6} $ , where i = 1, 2, 3, 4, 5, 6; \newline
And these two are indenpendent events, hence we can say,
\begin{align*}
	\pr{A|B} = \pr{B|A} = \pr{A}\pr{B}
\end{align*}
where we can assume A as event of tossing coin and B is event of tossing die \newline
\end{frame}

\section{Sum of numbers turned up is 3:}
\begin{frame}{Sum of numbers turned up is 3}
Since the coin can show only 1 or 6, the case of getting sum is possible only when coin turned up 1 and die turned up 2 \newline

Hence the Probability of getting sum of numbers that turned up as 3 is given by as follows,
\begin{align*}
	\pr{A = 1 | B = 2} &= \pr{A = 1} \pr{B = 2} \\
	&= \frac{1}{2} \times \frac{1}{6} \\
	&= \frac{1}{12}
\end{align*}
\begin{block}{}
Hence the required probablility is $\frac{1}{12}$
\end{block}
\end{frame}

\section{Sum of numbers turned up is 12:}
\begin{frame}{Sum of numbers turned up is 12}
Since the coin can show only 1 or 6, the case of getting sum is possible only when coin turned up 6 and die turned up 6 \newline

Hence the Probability of getting sum of numbers that turned up as 12 is given by as follows,
\begin{align*}
	\pr{A = 6 | B = 6} &= \pr{A = 6} \pr{B = 6} \\
	&= \frac{1}{2} \times \frac{1}{6} \\
	&= \frac{1}{12}
\end{align*}
\begin{block}{}
Hence the required probablility is $\frac{1}{12}$
\end{block}
\end{frame}

% Blocks frame 

\end{document}
