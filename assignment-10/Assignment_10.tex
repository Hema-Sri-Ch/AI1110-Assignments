%%%%%%%%%%%%%%%%%%%%%%%%%%%%%%%%%%%%%%%%%%%%%%%%%%%%%%%%%%%%%%%
%
% Welcome to Overleaf --- just edit your LaTeX on the left,
% and we'll compile it for you on the right. If you open the
% 'Share' menu, you can invite other users to edit at the same
% time. See www.overleaf.com/learn for more info. Enjoy!
%
%%%%%%%%%%%%%%%%%%%%%%%%%%%%%%%%%%%%%%%%%%%%%%%%%%%%%%%%%%%%%%%


% Inbuilt themes in beamer
\documentclass{beamer}

% Theme choice:
\usetheme{CambridgeUS}

% Packages
\usepackage{graphicx}
\graphicspath{{images/}}
\usepackage{gensymb}
\usepackage{amssymb}
\usepackage[cmex10]{amsmath}
\usepackage{amsthm}
\usepackage[export]{adjustbox}
\usepackage{bm}
\usepackage{longtable}
\usepackage{enumitem}
\usepackage{mathtools}
 \usepackage{tikz}
\usepackage[breaklinks=true]{hyperref}
\usepackage{listings}
\usepackage{color}                                            %%
\usepackage{array}                                            %%
\usepackage{longtable}                                        %%
\usepackage{calc}                                             %%
\usepackage{multirow}                                         %%
\usepackage{hhline}                                           %%
\usepackage{ifthen}                                           %%
\usepackage{lscape}     
\usepackage{multicol}
\usepackage{enumerate}
\DeclareMathOperator*{\Res}{Res}
\renewcommand\thesection{\arabic{section}}
\renewcommand\thesubsection{\thesection.\arabic{subsection}}
\renewcommand\thesubsubsection{\thesubsection.\arabic{subsubsection}}
\renewcommand\thesectiondis{\arabic{section}}
\renewcommand\thesubsectiondis{\thesectiondis.\arabic{subsection}}
\renewcommand\thesubsubsectiondis{\thesubsectiondis.\arabic{subsubsection}}
\hyphenation{op-tical net-works semi-conduc-tor}
\def\inputGnumericTable{}                                 %%
\lstset{
frame=single, 
breaklines=true,
columns=fullflexible
}

\newcommand{\BEQA}{\begin{eqnarray}}
\newcommand{\EEQA}{\end{eqnarray}}
\newcommand{\define}{\stackrel{\triangle}{=}}
\newcommand*\circled[1]{\tikz[baseline=(char.base)]{
    \node[shape=circle,draw,inner sep=2pt] (char) {#1};}}
%\bibliographystyle{IEEEtran}
\providecommand{\mbf}{\mathbf}
\providecommand{\pr}[1]{\ensuremath{\Pr\left(#1\right)}}
\providecommand{\qfunc}[1]{\ensuremath{Q\left(#1\right)}}
\providecommand{\sbrak}[1]{\ensuremath{{}\left[#1\right]}}
\providecommand{\lsbrak}[1]{\ensuremath{{}\left[#1\right.}}
\providecommand{\rsbrak}[1]{\ensuremath{{}\left.#1\right]}}
\providecommand{\brak}[1]{\ensuremath{\left(#1\right)}}
\providecommand{\lbrak}[1]{\ensuremath{\left(#1\right.}}
\providecommand{\rbrak}[1]{\ensuremath{\left.#1\right)}}
\providecommand{\cbrak}[1]{\ensuremath{\left\{#1\right\}}}
\providecommand{\lcbrak}[1]{\ensuremath{\left\{#1\right.}}
\providecommand{\rcbrak}[1]{\ensuremath{\left.#1\right\}}}
\theoremstyle{remark}
\newtheorem{rem}{Remark}
\newcommand{\sgn}{\mathop{\mathrm{sgn}}}
%\providecommand{\abs}[1]{\left\vert#1\right\vert}
%\providecommand{\res}[1]{\Res\displaylimits_{#1}} 
%\providecommand{\norm}[1]{\left\lVert#1\right\rVert}
%\providecommand{\norm}[1]{\lVert#1\rVert}
%\providecommand{\mtx}[1]{\mathbf{#1}}
%\providecommand{\mean}[1]{E\left[ #1 \right]}
\providecommand{\fourier}{\overset{\mathcal{F}}{ \rightleftharpoons}}
%\providecommand{\hilbert}{\overset{\mathcal{H}}{ \rightleftharpoons}}
\providecommand{\system}{\overset{\mathcal{H}}{ \longleftrightarrow}}
	%\newcommand{\solution}[2]{\textbf{Solution:}{#1}}
\newcommand{\solution}{\noindent \textbf{Solution: }}
\newcommand{\cosec}{\,\text{cosec}\,}
\providecommand{\dec}[2]{\ensuremath{\overset{#1}{\underset{#2}{\gtrless}}}}
\newcommand{\myvec}[1]{\ensuremath{\begin{pmatrix}#1\end{pmatrix}}}
\newcommand{\mydet}[1]{\ensuremath{\begin{vmatrix}#1\end{vmatrix}}}
\newcommand*{\permcomb}[4][0mu]{{{}^{#3}\mkern#1#2_{#4}}}
\newcommand*{\perm}[1][-3mu]{\permcomb[#1]{P}}
\newcommand*{\comb}[1][-1mu]{\permcomb[#1]{C}}
\numberwithin{equation}{subsection}
\makeatletter
\@addtoreset{figure}{problem}
\makeatother
\let\StandardTheFigure\thefigure
\let\vec\mathbf
\renewcommand{\thefigure}{\theproblem}
\def\putbox#1#2#3{\makebox[0in][l]{\makebox[#1][l]{}\raisebox{\baselineskip}[0in][0in]{\raisebox{#2}[0in][0in]{#3}}}}
     \def\rightbox#1{\makebox[0in][r]{#1}}
     \def\centbox#1{\makebox[0in]{#1}}
     \def\topbox#1{\raisebox{-\baselineskip}[0in][0in]{#1}}
     \def\midbox#1{\raisebox{-0.5\baselineskip}[0in][0in]{#1}}
\vspace{3cm}


% Title page details: 
\title{Assignment 10} 
\author{Hema Sri Cheekatla, CS21BTECH11013}
\date{\today}
\logo{\large \LaTeX{}}


\begin{document}
% Title page frame
\begin{frame}
    \titlepage 
\end{frame}

% Remove logo from the next slides
\logo{}


% Outline frame
\begin{frame}{Outline}
    \tableofcontents
\end{frame}

% Question frame
\begin{frame}{Questioin}
\section{Question}
    On observing 100 samples of x, find the 0.95 confidence interval of the median $x_{0.5}$ of x
\end{frame}

% Concept frame
\begin{frame}{Concept}

\section{Concept}
 \textbf{u Percentile:}\newline
    The u percentile of a random variable x is by definition a number $x_u$
such that $F(x_u) = u$. Thus $x_u$ is the inverse function $F^{-1}(u)$ of the distribution F(x)
of x.\newline

Let $y_k$ be the $k^{th}$ number assuming that all the numbers are arranged in ascending order\newline

$y_k < x_u < y_{k+r}$, iff atleast k and atmost $k+r-1$ of the samples $x_i$ are less than $x_u$

\end{frame}

\begin{frame}{Concept}
    The event $\cbrak{y_k < x_u < y_{k+r}}$ occurs iff number of success of event $\cbrak{x \leq x_u}$ in n repetitions of the experiment is atleast $k$ and atmost $k+r-1$.\newline

    And since $\pr{x \leq x_u} = u$ which means success probability $p=u$, we can obtain the following
    
    \begin{align}
        \pr{y_k < x_u < y_{k+r}} &= \sum_{m=k}^{k+r-1} \myvec{n \\ m} u^m (1-u)^{n-m} 
    \end{align}

\end{frame}

\begin{frame}{Concept}
    On approximation of above equation and assuming that X is a Normal or Gaussian Random Variable and n is large, for specific value of u,
    we obtain the following
    \begin{align}
        \pr{y_k < x_u < y_{k+r}} &= G(\frac{k + r - 0.5 - nu}{\sqrt{nu(1-u)}}) - G(\frac{k - 0.5 - nu}{\sqrt{nu(1-u)}}) = \gamma
     \end{align}
     Where, G is a Gaussian distribution function
     \begin{align}
         G(x) &= \int_{-\infty}^{x} \frac{1}{\sqrt{2\pi}} e^{-y^2/2} \, dy
     \end{align}
\end{frame}

\begin{frame}{Concept}

    From these equations, for a specific $\gamma$, r is minimum if nu is near the center of the interval $(k, k+r)$. This yeilds,
    \begin{align}
        k &\simeq nu-z_{1-\delta/2}\sqrt{nu(1-u)} \\
        k+r &\simeq nu+z_{1-\delta/2}\sqrt{nu(1-u)} 
    \end{align}
    
    where $z_u$ is the standard normal percentile with ,
\begin{align}
    u &= \frac{1}{\sqrt{2\pi}} \int_{-\infty}^{z_u} e^{-z^2/2} \, dz
\end{align}
\end{frame}

\begin{frame}{Solution:}

    \section{Solution:}
    Since we are supposed to find the median of $\vec{x}$ which is nothing but $x_{0.5}$ We have ,
    \begin{align}
        n &= 100 \\
        u &= 0.5 \\
        \gamma &= 1-\delta = 0.95 \\
        \implies \delta/2 &= 0.025\\
        \implies z_{1-\delta/2} &\simeq 2
    \end{align}
\end{frame}

\begin{frame}{Solution}

    Hence k and k+r are given by as follows,
    \begin{align}
        k &\simeq (100)(0.5) - 2(\sqrt{(100)(0.5)(1-0.5)}) \\
        k &= 40 \\
        \text{Similarly, } k+r &\simeq (100)(0.5) + 2(\sqrt{(100)(0.5)(1-0.5)}) \\
        \implies k+r &= 60
    \end{align}
    Therefore the median of $\vec{x}$ is inbetween $y_{40}$ and $y_{60}$\newline
    
    That is the median is inbetween 40th and 60th number.
\end{frame}
\end{document}

