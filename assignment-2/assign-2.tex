\let\negmedspace\undefined
\let\negthickspace\undefined
%\RequirePackage{amsmath}
\documentclass[journal,12pt,twocolumn]{IEEEtran}
%
% \usepackage{setspace}
 \usepackage{gensymb}
%\doublespacing
 \usepackage{polynom}
%\singlespacing
%\usepackage{silence}
%Disable all warnings issued by latex starting with "You have..."
%\usepackage{graphicx}
\usepackage{amssymb}
%\usepackage{relsize}
\usepackage[cmex10]{amsmath}
%\usepackage{amsthm}
%\interdisplaylinepenalty=2500
%\savesymbol{iint}
%\usepackage{txfonts}
%\restoresymbol{TXF}{iint}
%\usepackage{wasysym}
\usepackage{amsthm}
%\usepackage{pifont}
%\usepackage{iithtlc}
% \usepackage{mathrsfs}
% \usepackage{txfonts}
 \usepackage{stfloats}
% \usepackage{steinmetz}
 \usepackage{bm}
% \usepackage{cite}
% \usepackage{cases}
% \usepackage{subfig}
%\usepackage{xtab}
\usepackage{longtable}
%\usepackage{multirow}
%\usepackage{algorithm}
%\usepackage{algpseudocode}
\usepackage{enumitem}
 \usepackage{mathtools}
 \usepackage{tikz}
% \usepackage{circuitikz}
% \usepackage{verbatim}
%\usepackage{tfrupee}
\usepackage[breaklinks=true]{hyperref}
%\usepackage{stmaryrd}
%\usepackage{tkz-euclide} % loads  TikZ and tkz-base
%\usetkzobj{all}
\usepackage{listings}
    \usepackage{color}                                            %%
    \usepackage{array}                                            %%
    \usepackage{longtable}                                        %%
    \usepackage{calc}                                             %%
    \usepackage{multirow}                                         %%
    \usepackage{hhline}                                           %%
    \usepackage{ifthen}                                           %%
  %optionally (for landscape tables embedded in another document): %%
    \usepackage{lscape}     
% \usepackage{multicol}
% \usepackage{chngcntr}
%\usepackage{enumerate}

%\usepackage{wasysym}
%\newcounter{MYtempeqncnt}
\DeclareMathOperator*{\Res}{Res}
\DeclareMathOperator*{\equals}{=}
%\renewcommand{\baselinestretch}{2}
\renewcommand\thesection{\arabic{section}}
\renewcommand\thesubsection{\thesection.\arabic{subsection}}
\renewcommand\thesubsubsection{\thesubsection.\arabic{subsubsection}}

\renewcommand\thesectiondis{\arabic{section}}
\renewcommand\thesubsectiondis{\thesectiondis.\arabic{subsection}}
\renewcommand\thesubsubsectiondis{\thesubsectiondis.\arabic{subsubsection}}

% correct bad hyphenation here
\hyphenation{op-tical net-works semi-conduc-tor}
\def\inputGnumericTable{}                                 %%

\lstset{
%language=C,
frame=single, 
breaklines=true,
columns=fullflexible
}
%\lstset{
%language=tex,
%frame=single, 
%breaklines=true
%}
\begin{document}

%


\newtheorem{theorem}{Theorem}[section]
\newtheorem{problem}{Problem}
\newtheorem{proposition}{Proposition}[section]
\newtheorem{lemma}{Lemma}[section]
\newtheorem{corollary}[theorem]{Corollary}
\newtheorem{example}{Example}[section]
\newtheorem{definition}[problem]{Definition}
%\newtheorem{thm}{Theorem}[section] 
%\newtheorem{defn}[thm]{Definition}
%\newtheorem{algorithm}{Algorithm}[section]
%\newtheorem{cor}{Corollary}
\newcommand{\BEQA}{\begin{eqnarray}}
\newcommand{\EEQA}{\end{eqnarray}}
\newcommand{\define}{\stackrel{\triangle}{=}}
\newcommand*\circled[1]{\tikz[baseline=(char.base)]{
    \node[shape=circle,draw,inner sep=2pt] (char) {#1};}}
\bibliographystyle{IEEEtran}
%\bibliographystyle{ieeetr}
\providecommand{\mbf}{\mathbf}
\providecommand{\pr}[1]{\ensuremath{\Pr\left(#1\right)}}
\providecommand{\qfunc}[1]{\ensuremath{Q\left(#1\right)}}
\providecommand{\sbrak}[1]{\ensuremath{{}\left[#1\right]}}
\providecommand{\lsbrak}[1]{\ensuremath{{}\left[#1\right.}}
\providecommand{\rsbrak}[1]{\ensuremath{{}\left.#1\right]}}
\providecommand{\brak}[1]{\ensuremath{\left(#1\right)}}
\providecommand{\lbrak}[1]{\ensuremath{\left(#1\right.}}
\providecommand{\rbrak}[1]{\ensuremath{\left.#1\right)}}
\providecommand{\cbrak}[1]{\ensuremath{\left\{#1\right\}}}
\providecommand{\lcbrak}[1]{\ensuremath{\left\{#1\right.}}
\providecommand{\rcbrak}[1]{\ensuremath{\left.#1\right\}}}
\theoremstyle{remark}
\newtheorem{rem}{Remark}
\newcommand{\sgn}{\mathop{\mathrm{sgn}}}
%\providecommand{\abs}[1]{\left\vert#1\right\vert}
%\providecommand{\res}[1]{\Res\displaylimits_{#1}} 
%\providecommand{\norm}[1]{\left\lVert#1\right\rVert}
%\providecommand{\norm}[1]{\lVert#1\rVert}
\providecommand{\mtx}[1]{\mathbf{#1}}
%\providecommand{\mean}[1]{E\left[ #1 \right]}
\providecommand{\fourier}{\overset{\mathcal{F}}{ \rightleftharpoons}}
%\providecommand{\hilbert}{\overset{\mathcal{H}}{ \rightleftharpoons}}
\providecommand{\system}{\overset{\mathcal{H}}{ \longleftrightarrow}}
	%\newcommand{\solution}[2]{\textbf{Solution:}{#1}}
\newcommand{\solution}{\noindent \textbf{Solution: }}
\newcommand{\cosec}{\,\text{cosec}\,}
\providecommand{\dec}[2]{\ensuremath{\overset{#1}{\underset{#2}{\gtrless}}}}
\newcommand{\myvec}[1]{\ensuremath{\begin{pmatrix}#1\end{pmatrix}}}
\newcommand{\mydet}[1]{\ensuremath{\begin{vmatrix}#1\end{vmatrix}}}
%\numberwithin{equation}{section}
%\numberwithin{figure}{section}
%\numberwithin{table}{section}
%\numberwithin{equation}{subsection}
%\numberwithin{problem}{section}
%\numberwithin{definition}{section}
\makeatletter
\@addtoreset{figure}{problem}
\makeatother
\let\StandardTheFigure\thefigure
\let\vec\mathbf
%\renewcommand{\thefigure}{\theproblem.\arabic{figure}}
\renewcommand{\thefigure}{\theproblem}
%\setlist[enumerate,1]{before=\renewcommand\theequation{\theenumi.\arabic{equation}}
%\counterwithin{equation}{enumi}
%\renewcommand{\theequation}{\arabic{subsection}.\arabic{equation}}
\def\putbox#1#2#3{\makebox[0in][l]{\makebox[#1][l]{}\raisebox{\baselineskip}[0in][0in]{\raisebox{#2}[0in][0in]{#3}}}}
     \def\rightbox#1{\makebox[0in][r]{#1}}
     \def\centbox#1{\makebox[0in]{#1}}
     \def\topbox#1{\raisebox{-\baselineskip}[0in][0in]{#1}}
     \def\midbox#1{\raisebox{-0.5\baselineskip}[0in][0in]{#1}}
\vspace{3cm}


\title{
	AI1110 Assignment 2
}
\author{ Hema Sri Cheekatla, CS21BTECH11013% <-this % stops a space
}	

\maketitle
\begin{flushleft}
\textbf{Question 21a:}\newline
The cost function of a product is given by $C(x) = \frac{x^3}{3}-45x^2-900x+36$ where x is number of units produced. How many units should be produced to minimise the marginal cost?\newline
\textbf{Solution:}\newline
The marginal cost $MC = \frac{\Delta C}{\Delta Q}$ where,\newline
$\Delta C$ is change in cost\newline
$\Delta Q$ is change in quantity \newline
\begin{table}[h] 
\caption{\textbf{Table consisting symbols, formulae, description}}
\label{table:1}
\input{tables/table}
\end{table}

Hence Marginal Cost MC is given by,
\begin{align}
    MC &= \frac{d}{dx}\brak{C(x)} \\
    &= \frac{d}{dx}\brak{\frac{x^3}{3}-45x^2-900x+36}\\
    &= \frac{3x^2}{3}-2\times45x-900+0\\
    \implies MC &= x^2 -90x -900 
\end{align}
Now let $MC = y = f(x)$\newline
We need to find the number of units to be produced such that the Marginal cost is minimum.This Marginal cost $\brak{y=f(x)}$ is minimum for some $x=c$, at where it obeys the following conditions
\begin{enumerate}[label=(\roman*)]
    \item $\frac{dy}{dx}=0$ at x=c
    \item $\frac{d^2y}{dx^2}>0$ at x=c
\end{enumerate}
Now let us consider $\frac{dy}{dx}$
\begin{align}
    \frac{dy}{dx} &= \frac{d}{dx}\brak{x^2 -90x -900}\\
    \implies \frac{dy}{dx} &= 2x-90 =  f^{\prime}(x)
\end{align}
From first condition, at x=c, $\frac{dy}{dx}=0$
\begin{align}
    i.e., f^{\prime}(c) &= 0 \\
    2c-90 &= 0 \\
    \implies c &= 45 
\end{align}
Hence at x = 45, $\frac{dy}{dx}=0$\newline
Now let us consider the second condition at x=45
\begin{align}
    \frac{d^2y}{dx^2} &> 0 \\
    \frac{d}{dx}\brak{\frac{dy}{dx}} &> 0  \\
    \frac{d}{dx}\brak{2x-90} &> 0 \\
    2 &> 0
\end{align}
Hence the second condition is also valid at x = 45\newline
Therefore at $x = 45$ ,  $y=f(x)$ is minimum, i.e., Marginal cost is minimum if we produce 45 units.\newline

Hence the Number of units that are to be produced to minimise the marginal cost is 45 units.
\end{flushleft}
\end{document}