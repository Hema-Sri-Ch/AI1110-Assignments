\let\negmedspace\undefined
\let\negthickspace\undefined
%\RequirePackage{amsmath}
\documentclass[journal,12pt,twocolumn]{IEEEtran}
%
% \usepackage{setspace}
 \usepackage{gensymb}
%\doublespacing
 \usepackage{polynom}
%\singlespacing
%\usepackage{silence}
%Disable all warnings issued by latex starting with "You have..."
\usepackage{graphicx}
\graphicspath{{images/}}
\usepackage{amssymb}
%\usepackage{relsize}
\usepackage[cmex10]{amsmath}
%\usepackage{amsthm}
%\interdisplaylinepenalty=2500
%\savesymbol{iint}
%\usepackage{txfonts}
%\restoresymbol{TXF}{iint}
%\usepackage{wasysym}
\usepackage{amsthm}
%\usepackage{pifont}
%\usepackage{iithtlc}
% \usepackage{mathrsfs}
% \usepackage{txfonts}
 \usepackage{stfloats}
% \usepackage{steinmetz}
 \usepackage{bm}
% \usepackage{cite}
% \usepackage{cases}
% \usepackage{subfig}
%\usepackage{xtab}
\usepackage{longtable}
%\usepackage{multirow}
%\usepackage{algorithm}
%\usepackage{algpseudocode}
\usepackage{enumitem}
 \usepackage{mathtools}
 \usepackage{tikz}
% \usepackage{circuitikz}
% \usepackage{verbatim}
%\usepackage{tfrupee}
\usepackage[breaklinks=true]{hyperref}
%\usepackage{stmaryrd}
%\usepackage{tkz-euclide} % loads  TikZ and tkz-base
%\usetkzobj{all}
\usepackage{listings}
    \usepackage{color}                                            %%
    \usepackage{array}                                            %%
    \usepackage{longtable}                                        %%
    \usepackage{calc}                                             %%
    \usepackage{multirow}                                         %%
    \usepackage{hhline}                                           %%
    \usepackage{ifthen}                                           %%
  %optionally (for landscape tables embedded in another document): %%
    \usepackage{lscape}     
% \usepackage{multicol}
% \usepackage{chngcntr}
%\usepackage{enumerate}

%\usepackage{wasysym}
%\newcounter{MYtempeqncnt}
\DeclareMathOperator*{\Res}{Res}
\DeclareMathOperator*{\equals}{=}
%\renewcommand{\baselinestretch}{2}
\renewcommand\thesection{\arabic{section}}
\renewcommand\thesubsection{\thesection.\arabic{subsection}}
\renewcommand\thesubsubsection{\thesubsection.\arabic{subsubsection}}

\renewcommand\thesectiondis{\arabic{section}}
\renewcommand\thesubsectiondis{\thesectiondis.\arabic{subsection}}
\renewcommand\thesubsubsectiondis{\thesubsectiondis.\arabic{subsubsection}}

% correct bad hyphenation here
\hyphenation{op-tical net-works semi-conduc-tor}
\def\inputGnumericTable{}                                 %%

\lstset{
%language=C,
frame=single, 
breaklines=true,
columns=fullflexible
}
%\lstset{
%language=tex,
%frame=single, 
%breaklines=true
%}
\begin{document}

%


\newtheorem{theorem}{Theorem}[section]
\newtheorem{problem}{Problem}
\newtheorem{proposition}{Proposition}[section]
\newtheorem{lemma}{Lemma}[section]
\newtheorem{corollary}[theorem]{Corollary}
\newtheorem{example}{Example}[section]
\newtheorem{definition}[problem]{Definition}
%\newtheorem{thm}{Theorem}[section] 
%\newtheorem{defn}[thm]{Definition}
%\newtheorem{algorithm}{Algorithm}[section]
%\newtheorem{cor}{Corollary}
\newcommand{\BEQA}{\begin{eqnarray}}
\newcommand{\EEQA}{\end{eqnarray}}
\newcommand{\define}{\stackrel{\triangle}{=}}
\newcommand*\circled[1]{\tikz[baseline=(char.base)]{
    \node[shape=circle,draw,inner sep=2pt] (char) {#1};}}
\bibliographystyle{IEEEtran}
%\bibliographystyle{ieeetr}
\providecommand{\mbf}{\mathbf}
\providecommand{\pr}[1]{\ensuremath{\Pr\left(#1\right)}}
\providecommand{\qfunc}[1]{\ensuremath{Q\left(#1\right)}}
\providecommand{\sbrak}[1]{\ensuremath{{}\left[#1\right]}}
\providecommand{\lsbrak}[1]{\ensuremath{{}\left[#1\right.}}
\providecommand{\rsbrak}[1]{\ensuremath{{}\left.#1\right]}}
\providecommand{\brak}[1]{\ensuremath{\left(#1\right)}}
\providecommand{\lbrak}[1]{\ensuremath{\left(#1\right.}}
\providecommand{\rbrak}[1]{\ensuremath{\left.#1\right)}}
\providecommand{\cbrak}[1]{\ensuremath{\left\{#1\right\}}}
\providecommand{\lcbrak}[1]{\ensuremath{\left\{#1\right.}}
\providecommand{\rcbrak}[1]{\ensuremath{\left.#1\right\}}}
\theoremstyle{remark}
\newtheorem{rem}{Remark}
\newcommand{\sgn}{\mathop{\mathrm{sgn}}}
%\providecommand{\abs}[1]{\left\vert#1\right\vert}
%\providecommand{\res}[1]{\Res\displaylimits_{#1}} 
%\providecommand{\norm}[1]{\left\lVert#1\right\rVert}
%\providecommand{\norm}[1]{\lVert#1\rVert}
\providecommand{\mtx}[1]{\mathbf{#1}}
%\providecommand{\mean}[1]{E\left[ #1 \right]}
\providecommand{\fourier}{\overset{\mathcal{F}}{ \rightleftharpoons}}
%\providecommand{\hilbert}{\overset{\mathcal{H}}{ \rightleftharpoons}}
\providecommand{\system}{\overset{\mathcal{H}}{ \longleftrightarrow}}
	%\newcommand{\solution}[2]{\textbf{Solution:}{#1}}
\newcommand{\solution}{\noindent \textbf{Solution: }}
\newcommand{\cosec}{\,\text{cosec}\,}
\providecommand{\dec}[2]{\ensuremath{\overset{#1}{\underset{#2}{\gtrless}}}}
\newcommand{\myvec}[1]{\ensuremath{\begin{pmatrix}#1\end{pmatrix}}}
\newcommand{\mydet}[1]{\ensuremath{\begin{vmatrix}#1\end{vmatrix}}}
%\numberwithin{equation}{section}
%\numberwithin{figure}{section}
%\numberwithin{table}{section}
%\numberwithin{equation}{subsection}
%\numberwithin{problem}{section}
%\numberwithin{definition}{section}
\makeatletter
\@addtoreset{figure}{problem}
\makeatother
\let\StandardTheFigure\thefigure
\let\vec\mathbf
%\renewcommand{\thefigure}{\theproblem.\arabic{figure}}
\renewcommand{\thefigure}{\theproblem}
%\setlist[enumerate,1]{before=\renewcommand\theequation{\theenumi.\arabic{equation}}
%\counterwithin{equation}{enumi}
%\renewcommand{\theequation}{\arabic{subsection}.\arabic{equation}}
\def\putbox#1#2#3{\makebox[0in][l]{\makebox[#1][l]{}\raisebox{\baselineskip}[0in][0in]{\raisebox{#2}[0in][0in]{#3}}}}
     \def\rightbox#1{\makebox[0in][r]{#1}}
     \def\centbox#1{\makebox[0in]{#1}}
     \def\topbox#1{\raisebox{-\baselineskip}[0in][0in]{#1}}
     \def\midbox#1{\raisebox{-0.5\baselineskip}[0in][0in]{#1}}
\vspace{3cm}
\title{AI1110 Assignment 2}
\author{ Hema Sri Cheekatla, CS21BTECH11013}	
\maketitle
\begin{flushleft}
\textbf{Question 21a:}\newline
The cost function of a product is given by $C(x) = \frac{x^3}{3}-45x^2-900x+36$ where x is number of units produced. How many units should be produced to minimise the marginal cost?\newline
\textbf{Solution:}\newline
The marginal cost $MC = \frac{\Delta C}{\Delta Q}$ where,\newline
$\Delta C$ is change in cost\newline
$\Delta Q$ is change in quantity \newline
\begin{table}[h] 
\caption{\textbf{Table consisting symbols, formulae, description}}
\label{table:1}
\input{tables/table}
\end{table}

Hence Marginal Cost MC is given by,
\begin{align}
    MC &= \frac{d}{dx}\brak{C(x)} \\
    &= \frac{d}{dx}\brak{\frac{x^3}{3}-45x^2-900x+36}\\
    &= \frac{3x^2}{3}-2\times45x-900+0\\
    \implies MC &= x^2 -90x -900 
\end{align}
Now let $MC = y = f(x)$\newline
%We need to find the number of units to be produced \brak{c} such that the Marginal cost is minimum. 

We need to find the x at where f(x) is minimum by using gradient descent method.
Let us find $\nabla f(x)$
\begin{align}
    \frac{dy}{dx} &= \frac{d}{dx}\brak{x^2-90x-900} \\
   \implies f^{\prime}(x) &= 2x-90
\end{align}
We will be able to find the corresponding x value of the minimum of $f(x)$ by iterating the following equation till $\brak{f^{\prime}\brak{x_{k-1}}}$ approaches zero.
\begin{align}
x_{k} = x_{k-1} - \brak{\alpha\times f^{\prime}\brak{x_{k-1}}} 
\end{align}
where $x_{k-1}$ is initial assumed value/ previous obtained value \newline
$x_k$ is updated assumed value \newline
$\alpha$ represents the step size we are taking according to the slope $\brak{f^{\prime}\brak{x_{k-1}}}$

At first, let us randomly choose $x_{k-1}$ as $0$. Then, $f^{\prime}\brak{0} = -90 $.\newline
Since the slope is too far from zero and for manual purpose, we can take large step size. Hence let us choose $\alpha$ as $0.25$ \newline
Lets go through couple of iterations
\begin{align}
    x_k &= 0 - 0.25\times\brak{-90}  \nonumber \\
    &= 22.5 \\
    x_k &= 22.5 - 0.25\times\brak{-45} \nonumber \\
    &= 33.75 \\
    x_k &= 33.75 - 0.25\times\brak{-22.5} \nonumber \\
    &= 39.375 \\
    x_k &= 39.375 - 0.25\times\brak{-11.25} \nonumber \\
    &= 42.1875 \\
    x_k &= 42.1875 - 0.25\times\brak{-5.625} \nonumber \\
    &= 43.59375 \\
    x_k &= 43.59375 - 0.25\times\brak{-2.8125} \nonumber \\
    &= 44.296875 \\
    x_k &= 44.296875 - 0.25\times\brak{-1.40625} \nonumber \\
    &= 44.6484375 \\
    x_k &= 44.6484375 - 0.25\times\brak{-0.703125} \nonumber \\
    &= 44.82421875 \\
    x_k &= 44.82421875 - 0.25\times\brak{-0.3515625} \nonumber \\
    &= 44.912109375 \\
    x_k &= 44.912109375 - 0.25\times\brak{-0.17578125} \nonumber \\
    &= 44.9560546875 \\
    x_k &= 44.9560546875 - 0.25\times\brak{-0.087890625} \nonumber \\
    &= 44.978027344
\end{align}

Hence as the slope $f^{\prime}\brak{x_{k-1}}$ is tending to zero, $x_k$ is tending to 45. Hence the possible whole number at where the minimum of $f(x)$ exists is $c=45$.


\textbf{Note}: For solving the minimum using gradient descent method with algorithms, we can iterate through a lot of times to obtain the more precise value and we can take small step size too.\newline

Here is the corresponding graph obtained by using algorithm
\begin{figure}[h]
    \centering
    \includegraphics[width = \columnwidth]{fig_1.png}
    \caption{Finding minimum of $f(x)$ by gradient descent method}
    \label{fig:1}
\end{figure}

Therefore the number of units that are to be produced to get minimum marginal cost is 45 units.

%\end{flushleft}
\end{document}